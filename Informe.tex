\documentclass{article}
\usepackage{amsmath}
\usepackage{enumitem}
\usepackage{tcolorbox}



\begin{document}
\title{Informe de Simulación}
\author {Kevin Majim Ortega Alvarez \and Lidier Robaina Caraballo}
\maketitle

\section{Introducción}
\subsection{Breve descripciónn del proyecto}

Se desea crear una simulacion que simule un proceso de atencion  a varios clientes donde los servidores se encuantran en serie, o sea un cliente para
salir del sistema debe ser atendido por los n servidores. Cada servidor puede atender a un cliente o puede enviar a dicho cliente hacia otro servidor

\subsection{Objetivos y Metas}
Con dicha simulacion nos hemos propuesto responder las siguientes interrogantes:
\begin{enumerate}
\item{¿Cuánto tiempo le toma a un cliente completar su recorrido por los n servidores? ¿Qué pasa si añadimos mas servidores o aumentamos el tiempo que demora en cerrar el servicio?}
\item{¿Cuál es la media de salto para cada servidor? ¿Qué servidor envía mas personas a otros? ¿Si modificamos parametros distintos a la probabilidad de salto aumenta o disminuye?}
\item{¿Cuál es el tiempo promedio que demora un sistema en atender a todos los clientes? ¿Como se refleja el cambio si alteramos los parámetros?}\\
\end{enumerate}
Para dar respuestas a las anteriores preguntas tenemos como metas:
\begin{enumerate}
\item{Crear la simulación}
\item{Realizar el análisis estadístico de los resultados obtenidos de la simulación}
\item{Dar respuestas a nuestras interrogantes}
\item{Plantearnos nuevas interrogantes}
\end{enumerate}
\subsection{El sistema}
Los clientes llegan a un sistema que tiene n servidores, y las llegadas distribuye M. Cada cliente que llega debe ser atendido primero por el servidor 1 y, al completar el servicio en el servidor 1, el cliente pasa al servidor 2.

Cuando un cliente llega, entra en servicio con el servidor 1 si ese servidor está libre, o se une a la cola del servidor 1 en caso contrario. De manera similar, cuando el cliente completa el servicio en el servidor 1, entra en servicio con el servidor 2 si ese servidor está libre, o se une a su cola y asi sucesivamente. Después de ser atendido en el servidor n, el cliente abandona el sistema.

Los tiempos de servicio en el servidor i tienen la distribución Gi

Además del servidor i se tiene una probabilidad p de saltar al servidor j

\subsection{Variables que describen el sistema}
\begin{enumerate}
\item{Variable de tiempo: \textbf{time}}
\item{Variables contadoras:}
\begin{enumerate}[label=\alph*]
\item{\textbf{arrival number}: Cantidad de arribos en el tiempo}
\item{\textbf{out number}: Cantidad de salidas en el tiempo }
\item{\textbf{jump forward}: Cantidad de saltos hacia servidores más adelantados }
\item{\textbf{jump back}: Cantidad de saltos hacia servidores más atrasados }
\end{enumerate}
\item{Variables de estado del sistema}
\begin{enumerate}[label=\alph*]
\item{\textbf{serve's queue}: Lista que indica la cantidad de clientes en el servidor i}
\end{enumerate}
\item{Variables de salida}
\begin{enumerate}[label=\alph*]
\item{\textbf{arrival}: Lista con los diccionarios que representan el tiempo del cliente n el servidor i (donde i es el indice del servidor en la lista)}
\item{\textbf{wait time}: Tiempo que le tomo al cliente n salir del sistema (ser atendido por cada servidor)}
\item{\textbf{jump number}: Cantidad de saltos que se han producido desde el servidor i}
\end{enumerate}
\end{enumerate}

\section{Detalles de Implementacion}

El código está compuesto por dos archivos .py (Sim.py y main.py) donde en main se crean los parametros que se le introduciran a Sim.
El archivo Extract\_Samples.py se encarga de la extraccion de los datos mas relevantes de la simulación y de guardarlos cada uno en
documentos de excel. Ademas tenemos un archivo Statistic.py, el cual realiza los calculos estadisticos de los resultados que se encuentran en
los excel.\\
La función \textbf{Simulation} recibe como entrada:
\begin{enumerate}
\item{\textbf{n}: Cantidad de servidores $\rightarrow$ Int}
\item{\textbf{T}: Tiempo que se demora en cerrar la tienda $\rightarrow$ Int}
\item{\textbf{p}: Probabilidad con la que se desea que un servidor salte $\rightarrow$ Float}
\end{enumerate}
La probabilidad con la que un cliente sea enviado a otro servidor se genera a partir de una variable (change\_variable) $CH \sim \mathcal{U}(0,1)$, donde 
el parametro p representa que: para un $change\_variable \leq p$ el servidor atiende a su cliente y no lo envia hacia otro servidor en caso contrario, a continuacion tenemos el codigo:
\begin{tcolorbox}
if(change\_variable$>$p):\\
	jump\_number[serve]+=1\\
                while True:\\
                    change\_variable=np.random.uniform(0,n)\\
                    chan\_serve=int(change\_variable)\\
                    if chan\_serve==serve: continue\\

\end{tcolorbox}
Donde se vuelve a generar la variable a aleatoria $CH$, lo que ahora $CH\sim\mathcal{U}(0,n)$ para de esta forma convertirlo a entero y ese sera el servidor al que se envie el cliente. Si se envia a el mismo entonces se genera nuevamente.
\section{Resultados y Experimentos}
\subsection{Hallazgos de la simulación}

En la simulación se recogieron algunas variables de especial interés para dar respuestas a nuestras objetivos. Variables tales como:
\begin{enumerate}
\item{El tiempo que total que demora el modelo}
\item{Los tiempos que demoran cada cliente en recorrer los n servidores}
\item{La cantidad de saltos realizados por cada servidor}
\item{El tiempo en que un cliente es atendido por un servidor}
\end{enumerate}
También se extrajeron otras variables como: La cantidad de envíos a servidores más adelante y a servidores que se encuentran más atrasados dado que podemos preguntarnos tambien hacia donde van dirigidos los saltos con mayor frecuencia, y dado este caso podemos alterar el flujo para modelar situaciones más especificas. Una de las nuevas interrogantes sería: ¿Cuánto en promedio demora un servidor en atender a un cliente? Para responder está pregunta se extrajo también el tiempo en que cada cliente llega a un nuevo servidor, dado que, de esta manera se puede observar cual es el tiempo promedio que demora cada servidor en atender un cliente. Los que nos lleva a plantearnos preguntas como, ¿Qué servidor es más rápido?. Una interrogante que surge, casi al mismo tiempo con nuestro objetivo dos, es ¿hacía que servidores se envían más personas?.\\
Los datos recogidos en el modelo se presentan en tablas de trabajo de excel, a partir de las cuales se realizan los análisis estádisticos (tales como calculo de la media y de la desviacion muestral) los cuales se guardan en bloc de notas con el mismo nombre.

\subsection{Interpretación de los resultados}
\begin{table}[h]
\centering
\begin{tabular}{|c|c|}
\hline
\multicolumn{2}{|c|}{\textbf{Tabla de medias para $n=3,T=20,p=0.7$}} \\
\hline
Tiempo & 43.80582345727979\\
\hline
Tiempo de salida de un cliente & 25.482363177827487\\
\hline
Saltos para el servidor $n_1$ & 8.746666666666666\\
\hline
Saltos para el servidor $n_2$ & 8.793333333333333\\
\hline
Saltos para el servidor $n_3$ & 8.626666666666667\\
\hline
LLegada al servidor $n_1$ & 10.97679634360336\\
\hline
Llegada al servidor $n_2$ & 17.74840758330131\\
\hline
Llegada al servidor $n_3$ & 21.748768873390606\\
\hline
\end{tabular}
\label{tabla:ejemplo}
\end{table}

\begin{table}[h]
\centering
\begin{tabular}{|c|c|}
\hline
\multicolumn{2}{|c|}{\textbf{Tabla de medias para $n=3,T=20,p=0.8$}} \\
\hline
Tiempo & 38.82202390072639\\
\hline
Tiempo de salida de un cliente & 22.842904577624978\\
\hline
Saltos para el servidor $n_1$ & 4.9\\
\hline
Saltos para el servidor $n_2$ & 5.1\\
\hline
Saltos para el servidor $n_3$ & 5.373333333333333\\
\hline
LLegada al servidor $n_1$ & 11.012556104725936\\
\hline
Llegada al servidor $n_2$ & 16.203202188477857\\
\hline
Llegada al servidor $n_3$ & 19.73297639825384\\
\hline
\end{tabular}
\label{tabla:ejemplo}
\end{table}




















\end{document}















































